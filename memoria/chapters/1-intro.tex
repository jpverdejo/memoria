\chapter{Introducci\'on}
\label{cap:intro}

\section{Antecedentes y motivaci\'on}
\label{intro:motivacion}

Un grupo de científicos del Departamento de Física de la Universidad de Santiago de Chile trabaja en la simulación de los efectos del campo magnético en los átomos de distintos objetos, tomando en cuenta su forma, su material y su distribución atómica, entre otras características, usando una aplicación escrita en C empleando el Método de Monte Carlo \citep{MonteCarlo}.

No obstante lo anterior, uno de los grandes problemas de los científicos es que el proceso previo y posterior a la simulación es ``manual'', para definir el objeto deben hacer un dibujo en Microsoft Paint\textregistered, el cual es analizado por un \emph{script} de Matlab\textregistered\ que debe ser modificado para reflejar las características del objeto en específico que se quiere simular. Para generar imágenes para, por ejemplo, una publicación, también deben ejecutar ciertos \emph{scripts}, sin embargo esto es aún más complicado, ya que deben hacer ensayo y error hasta conseguir que la imagen sea representativa del resultado, puesto que muchas veces tienen errores de visualización que no reflejan el estado real. Estos dos sub-procesos hacen que el proceso de simulación sea tedioso, quitando mucho tiempo que podría ser usado en analizar los resultados.

Es aquí donde la informática puede contribuir, creando aplicaciones que mejoren estos procesos, que valoricen el tiempo de los científicos. Por consiguiente esta es una oportunidad única de ayudar a la obtención de conocimientos que permitan entender el entorno y de mejorar los procesos que permiten avanzar como sociedad hacia la comprensión del universo.


\section{Descripci\'on del problema}
\label{intro:problema}

Los científicos no disponen de una herramienta automatizada de diseño de objetos para aplicar la simulación MonteCarlo, que permita definir características físicas y geométricas y crear un archivo de entrada, que describa cada uno de los átomos, para la simulación, la cual es hecha por un software escrito en C por un anterior tesista del Departamento. Además se necesita que la herramienta permita mejorar el proceso de análisis y publicación de los resultados entregados por esta simulación.

\section{Estado del arte}

\subsection{Proceso actual}
Actualmente los investigadores deben preparar la simulación usando un \emph{script} en Matlab\textregistered, el cual analiza un archivo \emph{.bmp} creado en \emph{Microsoft Paint\textregistered}, esta imagen generalmente es pequeña, menor a 50px x 50px, y representa la primera capa del objeto (mirado desde arriba), para esto se marcan los pixeles que describen el objeto, de esta forma el mapa de \emph{bits} es binario, si un \emph{pixel} es negro se considera un 1, si es blanco se considera un 0. En este \emph{script} se describen características específicas del objeto que se quiere definir, entre ellos:
\begin{itemize}
	\item Número de capas: Cuantas veces se repite la primera capa para formar un objeto en 3D.
	\item Distribución de los átomos (ver anexo A): Como premisa se trabaja sólo con distribuciones cúbicas de átomos, no obstante estas distribuciones tienen ciertas características específicas, por ejemplo:
	\begin{itemize}
		\item Cúbico simple (SC por \emph{Simple Cubic}): En cada vértice de una distribución cúbica se encuentra un átomo.
		\item Centrado en las caras (FCC por \emph{Face-centered Cubic}): Además del átomo en cada vértice de la distribución cúbica, hay un átomo en el centro de cada cara de la distribución.
		\item Centrado (BCC por \emph{Body-centered Cubic}):  Además del átomo en cada vértice de la distribución cúbica, hay un átomo en el centro de cada distribución.
	\end{itemize}
	\item Coeficiente de escalamiento: Es un coeficiente usado para dar al objeto las medidas deseadas para ejecutar la simulación.
\end{itemize}

Este proceso produce un archivo de texto plano describiendo cada uno de los átomos del objeto sobre el cuál se aplicará la simulación.

Luego de ejecutada la simulación el software entrega múltiples archivos de texto, uno que define cada uno de los átomos con un ID y una posición en el espacio, y N archivos que definen la fuerza magnética de cada átomo en un tiempo dado. Los científicos deben seleccionar uno de estos archivos y aplicar un \emph{script} de Matlab\textregistered\ para poder visualizar el resultado.

El proceso de exportación de imágenes para publicaciones puede tomar un día de trabajo para los científicos, ya que deben hacer ``ensayo y error'' hasta que la imagen producida refleje lo que desean. Luego deben esperar por la aprobación por parte del profesor guía, en caso de ser rechazada, deben volver a ejecutar el proceso.

\subsection{Soluciones similares}
Existe una herramienta que hace simulaciones similares a las que hace el grupo de científicos llamada ``Go Parallel Magnet\textregistered''. A pesar de que la simulación no es exactamente la misma, el flujo de diseño es útil para el proyecto. Sin ir más lejos los científicos basan el proceso actual en éste.

``Go Parallel Magnet\textregistered'' usa para el diseño un sistema de multi-capas, donde se define la capa superior del objeto y se indica la cantidad de veces que ésta se repetirá. Con estos datos se crea un objeto en 3D que luego se transforma en la estructura molecular deseada.

\section{Propósitos de la solución}
\begin{enumerate}
  \item Mejorar el proceso de preparación de la simulación.
  \item Mejorar el proceso de exportación de imágenes para publicaciones.
\end{enumerate}


\section{Alcances o limitaciones de la solución}
\begin{itemize}
	\item El software se encargará del diseño de objetos para la simulación entregando la entrada para ésta y posteriormente de la visualización de los resultados, y de la exportación de estos para publicaciones, mas \textbf{NO} se encargará de la simulación en sí y queda fuera del alcance de la solución.
	\item La aplicación estará disponible para sistema operativo MAC OS X.
	\item El diseño de objeto será por capas, es decir, se define la ``vista superior'' y la cantidad de veces que se repetirá hacia abajo.
\end{itemize}

\section{Objetivos y alcance del proyecto}
\label{intro:objetivos}

\subsection{Objetivo general}
Crear un software que permita diseñar objetos en 3D, con características físicas y geométricas específicas, sobre los cuáles se aplicarán simulaciones de campo magnético a nivel atómico y analizar los resultados de forma visual, permitiendo la exportación de imágenes para publicaciones.

\subsection{Objetivos espec\'ificos}

Para la consecución del objetivo general, se plantean las siguientes metas intermedias para el software:

\begin{enumerate}
  \item Modelar el proceso de simulación del efecto de campo magnético en átomos efectuado por los científicos del Departamento de Física de la Universidad de Santiago de Chile.
  \item Diseñar la herramienta informática de ayuda para la investigación antes mencionada.
  \item Crear el software.
  \item Validar el cumplimiento de los requerimientos.
\end{enumerate}


\section{Características de la solución}

Como solución se propone la creación de un software que facilite el trabajo de los científicos. Esta aplicación se divide en dos funcionalidades:

\subsection{Diseño del objeto sobre el cuál se hará la solución:}

El software debe permitir crear visualmente un objeto en 3D, con ciertas características físicas como la distribución cúbica (ver anexo A). Luego de la creación y configuración del objeto la aplicación debe exportar un archivo de texto que sirva como entrada para el software que hará la simulación. Este archivo tiene un formato ya definido y debe describir la posición de cada átomo del objeto.

Este proceso se divide en tres características:

\begin{enumerate}
	\item Creación de objeto en base a un mapa de \emph{bits} binario, con características pre-definidas, y exportación de archivo de entrada para la simulación.
	\item Permitir la entrada de características del objeto, como número de capas y ordenamiento de los átomos.
	\item Visualización atómica en 3D del objeto sobre el cuál se hará la simulación.
\end{enumerate}

\subsection{Visualización de resultados de la simulación:}

El software debe tomar la totalidad de archivos de salida de la simulación como entrada y debe ser capaz de mostrar visualmente el estado magnético de cada átomo en un tiempo \emph{t}, también debe ser posible ver la simulación animada a través del tiempo, como un video.

La salida de la visualización serán imágenes en 2D del estado de la simulación en un tiempo \emph{t}, estas imágenes deben tener colores que permitan al lector entender el resultado a pesar de la dimensión faltante, por ejemplo, usando la proyección del vector en uno de los ejes y asignando un color según la intensidad de éste.

Este proceso se divide en tres características:

\begin{enumerate}
	\item Permitir ver el estado de la simulación en 3D en un tiempo \emph{t}.
	\item Permitir ver el cambio de estado de la simulación en 3D de forma animada.
	\item Permitir exportar el estado de la simulación en un tiempo \emph{t} en 2D para publicaciones.
\end{enumerate}


\section{Propósitos de la solución}
\begin{enumerate}
  \item Mejorar el proceso de preparación de la simulación.
  \item Mejorar el proceso de exportación de imágenes para publicaciones.
\end{enumerate}


\section{Alcances o limitaciones de la solución}
\begin{itemize}
	\item El software se encargará del diseño de objetos para la simulación entregando la entrada para ésta y posteriormente de la visualización de los resultados, y de la exportación de estos para publicaciones, mas \textbf{NO} se encargará de la simulación en sí y queda fuera del alcance de la solución.
	\item La aplicación estará disponible para sistema operativo MAC OS X.
	\item El diseño de objeto será por capas, es decir, se define la ``vista superior'' y la cantidad de veces que se repetirá hacia abajo.
\end{itemize}


\section{Metodolog\'ia y herramientas utilizadas}
\label{intro:metodologia}

\subsection{Metodolog\'ia}
Dado que se tiene conocimiento de los requerimientos mayores, pero pueden existir detalles al trabajar en un ámbito tan específico como simulaciones físicas, se decidió usar una modificación de Scrum \citep{SCRUM}.

Scrum está pensado para trabajar en equipos con varios desarrolladores, además de los cargos más de gestión, para esto se tienen 3 roles \citep{website:ScrumRoles}:

\begin{description}
  \item[\emph{Product Owner}] \hfill \\
  Es el encargado de maximizar el valor del trabajo el equipo, para esto tiene un alto conocimiento del producto mediante un contacto directo con los \emph{stakeholders} y facilita la comunicación de estos con el equipo de desarrollo, debido a esto es el responsable de decidir qué se va a construir, pero no el cómo. Para este proyecto el \emph{Product Owner} será el profesor Fernando Rannou, quién tiene contacto constante con los \emph{stakeholders} por proyectos paralelos que se están desarrollando.
  \item[\emph{Scrum Master}] \hfill \\
  Es el líder del equipo de desarrollo y debe tener un buen conocimiento de la metodología scrum, el cual debe traspasar al equipo de desarrolladores. Sus 3 principales tareas son: Guíar al equipo teniendo un conocimiento tanto del producto como de las tecnologías a utilizar, mantener al equipo avanzando eliminando toda dificultad que puedan tener durante el desarrollo, estas dificultades pueden ser tanto internas como externas, y enseñar a metodología \emph{scrum} al equipo. En este caso como solo un desarrollador trabajará en el proyecto y este tiene un gran conocimiento de la metodología gracias a sus años de experiencia laboral usandola, no se usará este rol.
  \item[Desarrollador] \hfill \\
  El desarrollador es el encargado de entregar los incrementales del producto, para eso se basa en la lista de tareas definidas por el \emph{Product Owner} al inicio de un \emph{sprint}. En este proyecto solo trabajará un programador.
\end{description}

Otras modificaciones hechas a la metodología fue modificar alguna de sus ceremonias, cambiando la reunión diaria (\emph{Daily Scrum}) por una semanal, entre el profesor y el desarrollador, las reuniones retrospectivas al finalizar cada \emph{sprint} se unió con la de planificación del siguiente periodo de desarrollo. Además, como es recomendado, se tuvo una reunión con los \emph{stakeholders} luego de cada sprint.

Estas modificaciones fueron necesarias para poder usar la metodología en un proyecto con solo un desarrollador y optimizando al máximo el tiempo usado en reuniones, debido al poco espacio en las agendas tanto del desarrollador como del \emph{Product Owner}.

\subsection{Herramientas de desarrollo}
\subsection{Modelado 3D: OpenGL}
Se usará OpenGL como biblioteca de modelado 3D, por ser el estado del arte en este ámbito, entre sus ventajas está el ser multi-plataforma, lo que eventualmente permitiría una rápida portación a otro sistema operativo, y el ser la más usada actualmente, lo que permite que tenga una amplia comunidad de usuarios que la soportan y documentan.

\subsection{Lenguaje de programación: Python}
Para el desarrollo se usará el lenguaje de programación Python con la biblioteca wxPython para Intefaz de Usuario. Esta biblioteca tiene soporte para la API OpenGL. Python, al ser un lenguaje multiplataforma, permitiría una rápida portación a otro sistema operativo en el futuro.

\subsection{Control de versiones: GIT}
Para el versionamiento del código se usará GIT, manteniendo un respaldo del repositorio con el código y la documentación en una máquina virtual con Linux ubicada en Estados Unidos.

\section{Ambiente de desarrollo}
Para el desarrollo se usará el siguiente ambiente de desarrollo:
\begin{itemize}
	\item Computador marca Apple, con una tarjeta gráfica que soporte OpenGL 3.2+ y sistema operativo Mac OS X para el desarrollo.
	\item Una máquina virtual con Linux, ubicada en Estados Unidos, para mantener un respaldo del código y de la documentación.
\end{itemize}
